\section{Conclusion}

These experiments simulated the performance of different TCP variants in networks with varying degrees of congestion. We analyzed the independent throughputs, latencies and drop rates of TCPs Tahoe, Reno, NewReno and Vegas. Further, we analyzed the performance of combinations of TCP variants. The dramatic performance differences between combinations such as TCP Vegas and TCP Reno/NewReno emphasize how important it is to be aware of the different variants on a single network. As we saw, when specifying TCP Vegas to one stream and TCP NewReno to another stream on the same network, the throughput of Vegas was drained almost completely, as NewReno continued to increase congestion window, forcing Vegas to back down. This compromised the fairness of afforded to the streams. On the other hand, multiple Vegas streams or multiple Reno streams, maintained fairness in the network. The fairness also depended on when the two streams were started. Starting the less aggressive TCP variant early helps mitigate some of the unfairness. 

Effects of queuing disciplines on Reno and SACK were also simulated. We found a combination of TCP variant and queuing algorithm (RED and SACK) that does not work well in terms of throughput. This was because of multiple RTOs. RED and DropTail presented a trade-off between throughput and latency. RED lowered the latency while lowering throughput. DropTail had higher latency but provided higher throughput as well. 

These experiments stress the importance of simulating and analyzing the performance of a specific network under various levels of congestion. The performance, measured in terms of end-to-end latency, throughput and drop ratr depend heavily on the kinds of TCP variants deployed and the queuing algorithms. Simulation helps us assess if a specific network will achieve the desired performance metrics and provide fairness. Choosing an inefficient TCP variant, an unfair combination of TCP variants or a queuing algorithm which does not meet the needs of the type of data exiting the network, can be detrimental to a network. These network configurations must be carefully analyzed whether it be in a corporate, government or academic network. Achieving a fair share of bandwidth in an environment where different TCP variants and transport protocols are used is difficult. 

On a final note, our simulations showed that TCP NewReno maintained high throughput on its own as well as when combined with other variants. However, it would be interesting to extend these tests to further study situations in which the immediate retransmision of packets caused by partial acknowledgments were only adding congestion to a network. It's possible that partial acknowledgments do not signify packet loss. In this case retransmission is not necessary and is only adding more congestion to a network.