\section{Experiment 3: Influence of Queuing}

This experiment analyzes the performance of TCP Reno and TCP Sack using the queuing algorithms: DropTail and RED. TCP Sack, increases and decreases the congestion window the same as TCP Reno does, when dealing with the first packet loss. Upon the first sign of packet loss, both variants cut the congestion window in half, and proceed to retransmit the lost packet. 

In situations involving multiple packet losses, TCP Sack waits until the number of outstanding, or yet to be acknowledged packets, is less than the congestion window. While in fast recovery, TCP Sack keeps a record of outstanding packets as well as a corresponding counter. Reno only has the foresight to keep track of a single lost packet. Sack uses partial acknoweldgements to decrease the counter of outstanding packets at a rate which increases the congestion window faster than slowstart, which is why the performance of Sack benefits from a slightly higher throughput and a slightly lower latency than Reno.

DropTail fills up the queue, and drops any further packets when the queue is full. It is first come, first serve. Random Early Detection (RED) accepts all packets when the queue is below some threshhold, and randomly accepts packets when the queue is above the threshhold but still less then the maximum capacity. This allows it to reduce the power of large and fast packet burst which would otherwise monoplize the bandwidth and fill up the queue (in the case of DropTail).

Our experiment held constant a CBR flow rate of 8Mbps and a packet size of 4,000 bytes, which cause congestion and packet loss in the network. The results of the experiment showed an average throughput using RED in TCP Reno and SACK was .214 Mbps and .225 Mbps, respectively. The average throughput using the DropTail queuing algorithm in Reno and SACK, were .148 Mbps and .150 Mbps, respecitvely. The increased throughput using the RED queuing algorithm is because the network is able to maintain a more steady flow of traffic, and avoid being under or over utilized. Random dropping however, did double the drop rate - because packets were dropped when they were above some threshold to encourage fairness among the tcp traffic. TCPs Sack and Reno experienced 118 and 116 dropped packets using DropTail and experienced 216 and 267 dropped packets using RED.

The combination of TCP Sack using the RED queuing algorithm, consisted of the highest throughput, lowest average and end-to-end latencies of the four combinations of variants and queuing algorithm. As previously mentioned, it must be noted, there was a relatively high number of drop packets in comparison to results computed when using DropTail. It comes down to the amount of congestion on the network. If there is moderate congestion, it is efficient to use TCP Sack and RED together. If there is a rate of congestion, far exceeding the threshhold using in RED, at which point, packets are randomly dropped, it may be less efficient to use this queuing algorithm.