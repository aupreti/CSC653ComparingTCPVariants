\section{Experiment 2: Analyzing Fairness between TCP Variants}

This experiment used two TCP flows and a single CBR flow, to analyze the fairness, between the TCP flows over different pairs of variants. We tested different combinations of variants, including assigning TCP Vegas to both flows, TCP Reno to both flows, NewReno to one flow and Reno to another, and NewReno to one flow and Vegas to another. Further, we ran these tests twice. The first test analyzes fairness in a network with a CBR flow rate of 1Mbps and a packet size of 1000 Bytes. The second test analyzes fairness is a network with a CBR flow rate of 10Mbps and a packet size of 4000 bytes. The first test simulates a network with no congestion, and the second test simulates a network with congestion and packet loss.

In the network with no congestion (test 1), the only combination of TCP flows which did not exhibit the same throughput rates, was that of TCP NewReno and TCP Vegas. TCP Vegas had a higher throughput than TCP NewReno, similarly to what we found in experiment 1. Because there was no congestion on the network, the flows acted as they would independenltly of each other. The combinations of NewReno/Reno and Reno/Reno, experienced identical throughputs, latencies, etc. because there was no packet loss, so they essentially followed the same protocol in this simulation. Vegas/Vegas also split the throughputs evenly, because they both follow the same protocol, however their throughputs were significantly lower than any other combination we tested, again consistent with what we found in experiment 1.

Given the combination of TCPs NewReno and Vegas, the Vegas stream suffers a lower bandwidth. Vegas senses congestion on a network by load of the queue, so it avoids filling the queue. NewReno, on the other hand, fills up the queue and in turn is afforded an amount of bandwidth proporiional to the amount of the queue it filled up.

