\section{Conclusion}

These experiments simulated the performance of different TCP variants in networks with varying degrees of congestion. We analyzed the independent throughputs, latencies and drop rates of TCPs Tahoe, Reno, NewReno and Vegas. Further, we analyzed the performance of combinations of TCP variants. The dramatic performance differences between combinations such as TCP Vegas and TCP Reno/NewReno emphasize how important it is to be aware of the different variants on a single network. As we saw, when specifying TCP Vegas to one stream and TCP NewReno to another stream on the same network, the throughput of Vegas was drained almost completely, as NewReno continued to increase congestion window, forcing Vegas to backdown. This compromised the fairness of afforded to the streams. On the other hand, multiple Vegas streams or multiple Reno streams, maintained fairness in the network.

***The last experiment...


These experiments stress the importance of analyzing a specific network, the amount of congestion that is normal and the importance of different data reaching their destinations and the speed at which it must be received. Choosing an inefficient TCP variant, an unfair combination of TCP variants or a queing algorithm which does not meet the needs of the type of data exiting the network, can be detreimnetal to a network. These network configurations must be carefully analyzed whether it be in a corporate, government or academic network.