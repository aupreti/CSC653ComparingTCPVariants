\section{Experiment 2: Analyzing Fairness between TCP Variants}

This experiment used two TCP flows and a single CBR flow, to analyze the fairness between the TCP flows among different pairs of TCP variants. We tested different combinations of variants, including assigning TCP Vegas to both flows, TCP Reno to both flows, NewReno to one flow and Reno to another, and NewReno to one flow and Vegas to another. Further, we ran two slightly different simulations. The first simulation analyzes fairness in a network with a CBR flow rate of 1Mbps and a packet size of 1000 Bytes. The second simulation analyzes fairness is a network with a CBR flow rate of 10Mbps and a packet size of 10,000 bytes. The first test simulates a network with little to no congestion, and the second test simulates a network with congestion and packet loss.

\subsection{Simulation 1: Little-to-No Network Congestion}

In the network with little congestion, the throughputs of the individual streams given the combinations of Reno/Reno and NewReno/Reno were all equal, at .56 Mbps. This is to be expected. There were no packet losses, the NewReno stream acted as a Reno stream. This is because the variants only differ in what they do after multiple packet drops. After the first packet drop, both variants use fast retransmit. Afterwards, Reno enters fast recovery and cuts its congestion window in half while NewReno inflates its congestion window to allow for outstanding packets between the source and destination to complete their transmissions. The throughputs of the two Vegas streams, were .55 and .6 Mbps. This can be a result of one stream calculating a higher than expected RTT and reducing the congestion window, allowing the other to maintain and increase the cwnd size. Unlike Reno and NewReno, Vegas is not deterministic given no packet drops. The largest discrepancy between any two streams we tested, was that of NewReno and Vegas. The average throughput of NewReno was .98 Mbps, and the throughput of Vegas was .15 Mpbs. 

TCP Vegas senses congestion through larger than expected RTT values, and proceeds to decrease its congestion window. TCP NewReno continuously increases its window size until loss actually occurs. Even though there were no packet drops and little to no congestion, the aggressive nature of TCP NewReno monopolizes the bandwidth of the streams, reducing the throughput of the more cautious TCP Vegas.

\subsection{Simulation 2: High Network Congestion}

Each combination of variants in the second simulation saw high amounts of congestion. Although the combination of TCP Vegas, once again resulted in similar throughputs for individual streams - at about .1 Mbps each, we note that these throughputs are much smaller than what was sccen in th eprevious simulation. The throughput was much lower because each stream sensed the congestion caused by the other flow; however, there were no packet drops, instead the congestion window was continiously decreased. The combination of Reno/Reno also saw lower, but equal throughputs, with each stream maintaining an average throughput of about .2 Mbps. In this case, we saw about 23 packet drops. Still, the fast recovery and fast retransmit process is the same for both of these streams, so the similar throughputs are to be expected.

The combination of NewReno/Reno resulted in a larger throughput discrepency given the congested network. NewReno, as expected achieved the slightly higher throughput of .23 Mbps where as Reno's throughput was .185 Mbps. There were 30 total packets dropped. If the simulation had run for longer, we predict the number of dropped packets would increase and the discrepency between the throughputs of the NewReno and Reno streams would grow even larger.
The reason for this discrepency is that NewReno will inflate its congestion window size by the number of duplicate acknowledgments it received, to facilitate the transmission of the remaining packets going between the source and destination. Ultimately, NewReno deflates its congestion window, but only after outstanding packets have been properly acknowledged by the receiver. Reno, on the other hand, will decrease its congestion window by two, after having completed fast retransmit. This is done with the assumption that size of packets entering the network is too big and decreasing the window size will reduce the amount of congestion.

This discrepency reappears more jurassicaly when assigning one stream to TCP NewReno and another to TCP Vegas. NewReno had an average throughput of .26 Mpbs and Vegas at .1 Mpbs. Only 8 packets were dropped in this simulation, all by the TCP NewReno stream. NewReno utilizes an increasing amount of the link before and in the immediate aftermath of a packet loss occur. Vegas, on the other hand, continuously decreases its congestion window as it perceives the increased utilization by NewReno, as congestion. However, we notice the discrepency in throughputs here, is not as large as the discrepency we saw in the first simulation. This is because the packets dropped by NewReno, although inflating the congestion window intiially, ultimately deflated the window, reducing the size of the packets being released into the network.




